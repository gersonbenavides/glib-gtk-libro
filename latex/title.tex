\begin{titlepage}
    \begin{center}
        \newpagecolor{backcolour}\afterpage{\restorepagecolor}
        
        \vspace*{0.64cm}
        
        \textsc{
            \Huge La plataforma de desarrollo\\[0.32cm]
            GLib/GTK
        }
        
        \vspace{0.64cm}
        
        \textrm{\Large Una guía de introducción}
        
        \vspace{1.28cm}
        
        \includegraphics[width=0.64\textwidth]{logo-gtk.png}
        
        \vspace{1.92cm}

        \texttt{\large Versión \bookversion}

        \vfill
            
        {\large Sébastien Wilmetn | Gerson Benavides}
    \end{center}
\end{titlepage}

\mbox{}

\vfill

\section{Licencia}
    \label{intro-license}
    
    \begin{center}
        \includegraphics[height=0.8cm]{assets/img/creative-commons.pdf}
    \end{center}
    
    Este trabajo está autorizado bajo una licencia internacional Creative Commons Attribution-ShareAlike 4.0:
    
    \url{https://creativecommons.org/licenses/by-sa/4.0/}
    
    \vspace{0.16cm}
    
    Algunas secciones están basadas en el libro \emph{GTK+/Gnome Application Development}, escrito en 1999, editado por New Riders Publishing y con licencia de Open Publication License. La última versión de la licencia de publicación abierta se puede encontrar en:
    
    \url{http://www.opencontent.org/openpub/}

\newpage

\section{Prefacio}
    Este texto es una guía de inicio para dar comienzo con la plataforma de desarrollo GLib/GTK, haciendo uso del lenguaje C. El libro nace en 2016 como un proyecto personal de \href{https://louvilug.tuxfamily.org/swilmet/}{Sébastien Wilmetn}, el cual desarrolla hasta el año 2019. El proyecto es luego retomado por \href{https://gersonbdev.github.io/about/}{Gerson Benavides} a partir del año 2021.
    
    En ocasiones se asumirá que el lector usa un sistema similar a Unix, no obstante, la mayor parte de lo escrito en este libro es aplicable a otros sistemas operativos.
    
    Tenga en cuenta que este libro está lejos de estar terminado, en este momento esta leyendo la versión \bookversion. Si tiene algún comentario, no dude en ponerse en contacto escribiendo a \href{mailto:gersonbdev@gmail.com}{gersonbdev@gmail.com}.
    
    El código fuente y los diferentes formatos del libro se pueden encontrar en los siguientes enlaces:
    
    \begin{itemize}
        \item Español
            \begin{itemize}
                \item Web: \url{https://gersonbdev.github.io/glib-gtk-libro/}
                \item PDF: \url{https://raw.githubusercontent.com/gersonbdev/glib-gtk-libro/master/glib-gtk-libro.pdf}
                \item Repositorio: \url{https://github.com/gersonbdev/glib-gtk-libro}
            \end{itemize}
        \item Inglés
            \begin{itemize}
                \item Web: \url{https://people.gnome.org/~swilmet/glib-gtk-book/}
                \item PDF: \url{https://people.gnome.org/~swilmet/glib-gtk-dev-platform.pdf}
                \item Repositorio: \url{https://github.com/swilmet/glib-gtk-book}
            \end{itemize}
    \end{itemize}


\section{Agradecimientos}
    Gracias a: Christian Stadelmann, Errol van~de~l'Isle, Andrew Colin Kissa y Link Dupont.

\newpage


\section{Contribución}

    Si desea contribuir con el desarrollo de este libro puede apoyar con:
    
    \subsection{Soporte en redacción}
    Escribiendo o revisando el texto, para hacerlo participe en el repositorio del libro (\url{https://github.com/gersonbdev/glib-gtk-libro}) dando su opinión o desarrollando las tareas propuestas (revise el archivo TODO).
    
    \subsection{Soporte financiero}
    El libro se publica como un documento \emph{libre} y es gratuito. Pero no se materializa en un espacio vacío, se necesita tiempo para escribir. Al donar, demuestra su aprecio por este trabajo y ayuda a su desarrollo futuro.
    
    Puede encontrar un botón de donación en:
    
    \url{https://gersonbdev.github.io/about/}
    
    ¡Gracias!
    