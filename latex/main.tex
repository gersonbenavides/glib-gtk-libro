\documentclass[a4paper,11pt]{report}
\usepackage[T1]{fontenc}
\usepackage[utf8]{inputenc}
\usepackage[spanish]{babel}
\usepackage{lmodern}
\usepackage{graphicx}
\usepackage{url}
\usepackage{parskip}
\usepackage{pmboxdraw}

% Margenes
\usepackage[top=2cm, bottom=3cm, left=3.8cm, right=3.8cm]{geometry}

% Propiedades y vínculos de archivos PDF
\usepackage{hyperref}
\hypersetup{
  pdfauthor = {Sébastien Wilmet},
  pdftitle = {La plataforma de desarrollo GLib/GTK},
  pdfcreator = {Texlive},
  pdfproducer = {Texlive},
  %colorlinks = false,
  pdfborder = 0 0 0
}

% Se acuña otro paquete para resaltar la sintaxis de código. Por defecto, acuñado tiene mejores colores para leer el documento en una pantalla, pero creo que es posible tener colores decentes con listados también, solo requiere más configuración. Para fines de impresión (sin colores), creo que los listados son mejores. Y este documento está destinado a ser impreso.
\usepackage{listings}
\lstset{
  language = C,
  basicstyle = {\small\ttfamily},
  basewidth = 0.5em,
  showstringspaces = false,
  captionpos = b
%   floatplacement = htbp
% Los marcos se ven bien, pero deben usarse con precaución. Un marco alrededor de algo significa que el "algo" es importante. Quizás se utilice un marco más adelante en este libro para dar un resumen de cosas importantes. Si cada fragmento de código está rodeado por un marco, el texto importante no estará tan resaltado como debería.
%   frame = single,
%   frame = tb,
% Será útil más adelante para una gran parte de código (cree un nuevo entorno llamado lstlistinglarge o algo así).
%   linewidth = 15cm,
%   xleftmargin = -1cm,
%   xrightmargin = -1cm
}

\lstloadlanguages{C, C++, Lisp, bash}

% Separe los listados del texto (recomendado en el documento de listados, en lugar de usar marcos).
% Pero con subtítulos en la parte inferior, no es muy útil.
%\newcommand{\topfigrule}{\hrule\kern-0.4pt\relax}
%\newcommand{\botfigrule}{\hrule\kern-0.4pt\relax}

\newcommand{\shellcmd}[1]{\texttt{#1}}

\newcommand{\bookversion}{0.8}

\title{La plataforma de desarrollo GLib/GTK\\[0.3cm]
{\normalsize Una guía de introducción}\\[0.3cm]
{\normalsize Versión \bookversion}}

\author{Sébastien Wilmetn}

\begin{document}

\maketitle
\tableofcontents

\include{content/00-intro}
\include{content/01-glib}
\part{Programación orientada a objetos en C\label{oop}}

\chapter*{Introducción a la Parte II}
\setcounter{footnote}{0}

Ahora que está familiarizado con la biblioteca principal de GLib, ¿cuál es el siguiente paso? Como se explicó en la sección Learning Path (sección~\ref{intro-learning-path} p.~\pageref{intro-learning-path}), el seguimiento lógico es la programación orientada a objetos (OOP) en C y los conceptos básicos de GObject.

Cada widget GTK es una subclase de la clase base GObject. Entonces, conocer los conceptos básicos de GObject es importante para \emph{usar} un widget GTK u otra utilidad basada en GObject, pero también para \emph{crear} tus propias clases de GObject.

Es importante notar que aunque el lenguaje C no está orientado a objetos, es posible escribir código C ``semi-orientado a objetos'' fácilmente, sin GObject. Para fines de aprendizaje, con eso comienza esta parte. GObject es entonces más fácil de aprender. Lo que GObject agrega son más características como recuento de referencias, herencia, funciones virtuales, interfaces, señales y más.

Pero, ¿por qué seguir un estilo orientado a objetos en primer lugar? Un código orientado a objetos permite evitar variables globales. Y si ha leído algún tipo de guía de mejores prácticas de programación, sabe que \emph{debería}, si es posible, evitar las variables globales \footnote{Una variable global en C puede ser una variable \lstinline{static} declarada en el parte superior de un archivo * .c, al que se puede acceder desde cualquier función en ese archivo * .c. Esto a veces es útil, pero debe evitarse si es posible. Hay otro tipo de variable global en C: una variable \lstinline{extern} a la que se puede acceder desde cualquier archivo *.c. Este último es mucho peor que el primero.}. Porque el uso de datos globales hace que el código sea más difícil de administrar y comprender, especialmente cuando un programa se vuelve más grande. También hace que el código sea más difícil de reutilizar. En cambio, es mejor dividir un programa en partes más pequeñas e independientes, de modo que pueda concentrarse solo en una parte del código a la vez.

Esta parte del libro consta de dos capítulos:
\begin{itemize}
    \item Capítulo~\ref{oop-semi}, que explica cómo escribir sus propias clases semi-OOP;
    \item Capítulo~\ref{oop-gobject}, que explica los conceptos básicos de GObject.
\end{itemize}
\include{content/02-oop-semi}
\include{content/02-oop-gobject}
\include{content/03-gtk-app-arch}
\part{Lectura adicional \label{further-reading}}

\chapter{Lecturas adicionales}

En este punto, debe conocer los conceptos básicos de GLib core y GObject. No necesitas saber \ emph {todo} sobre GLib core y GObject para continuar, pero tener al menos un conocimiento básico te permitirá aprender más fácilmente GTK y GIO, o cualquier otra biblioteca basada en GObject.

\section{GTK y GIO}
GTK y GIO se pueden aprender en paralelo.

Debería poder usar cualquier clase de GObject en GIO, solo lea la descripción de la clase y hojee la lista de funciones para tener una descripción general de las características que proporciona una clase. Entre otras cosas interesantes, GIO incluye:
\begin{itemize}
  \item \lstinline{GFile} para manejar archivos y directorios.
  \item \lstinline{GSettings} para almacenar la configuración de la aplicación.
  \item \lstinline{GDBus}: una API de alto nivel para el sistema de comunicación entre procesos D-Bus.
  \item \lstinline{GSubprocess} para iniciar procesos secundarios y comunicarse con ellos de forma asincrónica.
  \item \lstinline{GCancellable}, \lstinline{GAsyncResult} y \lstinline{GTask} para usar o implementar tareas asincrónicas y cancelables.
  \item Muchas otras funciones, como flujos de E/S, soporte de red o soporte de aplicaciones.
\end{itemize}

Para crear aplicaciones gráficas con GTK, no se preocupe, la documentación de referencia tiene una guía de introducción, disponible con Devhelp o en línea en: \\
\url{https://developer.gnome.org/gtk3/stable/}

Después de leer la guía de introducción, lea toda la referencia de la API para familiarizarse con los widgets, contenedores y clases base disponibles. Algunos widgets tienen una API bastante grande, por lo que también están disponibles algunos tutoriales externos, por ejemplo, para \lstinline{GtkTextView} y \lstinline{GtkTreeView}. Consulte la página de documentación en: \\
\url{http://www.gtk.org}

También hay una serie de pequeños tutoriales sobre varios temas GLib / GTK: \\
\url{https://wiki.gnome.org/HowDoI}

\section{Escribir sus propias clases de GObject}

El capítulo~\ref{oop-gobject} explica cómo \emph{usar} una clase GObject existente, que es muy útil para aprender GTK, pero no explica cómo \emph{crear} tus propias clases GObject. Escribir sus propias clases de GObject permite contar con referencias, puede crear sus propias propiedades y señales, puede implementar interfaces, anular funciones virtuales (si la función virtual no está asociada a una señal), etc.

Como se explicó al principio del capítulo~\ref{oop-gobject}, si desea obtener información más detallada sobre GObject y saber cómo crear subclases, la documentación de referencia de GObject contiene capítulos introductorios: ``\emph{Concepts}'' y ``\emph{Tutorial}'', disponibles como de costumbre en Devhelp o en línea en: \\
\url{https://developer.gnome.org/gobject/stable/}

\section{Sistema de compilación}

Un Makefile básico generalmente no es suficiente si desea instalar su aplicación en diferentes sistemas. Por tanto, se necesita una solución más sofisticada. Para un programa basado en GLib/GTK, existen dos alternativas principales: Autotools y Meson.

GNOME y GTK históricamente usan Autotools, pero a partir de 2017 algunos módulos (incluido GTK) están migrando a Meson. Para un nuevo proyecto, se puede recomendar Meson.

\subsection{Las herramientas automáticas}

Las Autotools comprenden tres componentes principales: Autoconf, Automake y Libtool. Está basado en scripts de shell, macros m4 y \shellcmd{make}.

Las macros están disponibles para varios propósitos (la documentación del usuario, estadísticas de cobertura de código para pruebas unitarias, etc.). El libro más reciente sobre el tema es \emph{Autotools}, de John ~ Calcote \cite{autotools}.

Pero las Autotools tienen la reputación de ser difíciles de aprender.

\subsection{Meson}

Meson es un sistema de construcción bastante nuevo, es más fácil de aprender que Autotools y también resulta en construcciones más rápidas. Algunos módulos de GNOME ya usan Meson. Consulte el sitio web para obtener más información:\\
\url{http://mesonbuild.com/}

\section{Mejores prácticas de programación}

Se recomienda seguir las Pautas de programación de GNOME~\cite{gnome-programming-guidelines}.

La siguiente lista no tiene nada que ver con el desarrollo de GLib/GTK, pero es útil para cualquier proyecto de programación. Después de tener algo de práctica, es interesante aprender más sobre las \emph{mejores} prácticas de programación. Escribir código de buena calidad es importante para prevenir errores y para mantener una pieza de software a largo plazo.

\begin{itemize}
  \item \emph{El} libro sobre las mejores prácticas de programación es \emph{Código completo}, de Steve~McConnell \cite{code-complete}. Muy recomendable \footnote{Aunque el editor de \emph{Código completo} es Microsoft Press, el libro no está relacionado con Microsoft o Windows. El autor a veces explica cosas relacionadas con el código abierto, UNIX y Linux, pero uno puede lamentar la ausencia total de la mención ``software libre/free'' y todos los beneficios de la libertad, en particular para este tipo de libros: poder aprender leyendo el código de otros. Pero si está aquí, es de esperar que ya sepa todo esto.}.

  \item Para obtener pautas sobre POO específicamente, consulte \emph{Heurística de diseño orientado a objetos}, de Arthur~Riel \cite{oop-book}.

  \item Una excelente fuente de información es la web de Martin ~ Fowler: refactorización, metodología ágil, diseño de código, ...\\
  \url{http://martinfowler.com/}
\end{itemize}

Más relacionados con GNOME, los artículos de Havoc ~ Pennington tienen buenos consejos que vale la pena leer, incluidos ``\emph{Trabajando en software libre}'', ``\emph{Interfaz de usuario de software libre}'' y ``\emph{Mantenimiento de software libre : Adición de funciones}'':\\
\url{http://ometer.com/writing.html}
\begin{thebibliography}{10}
\addcontentsline{toc}{chapter}{Bibliografía}

\bibitem{k-r-book}
Brian \textsc{Kernighan} y Dennis \textsc{Ritchie}
(1988).\\
\emph{The C Programming Language}
(2da ed.). Prentice Hall.

\medskip
\bibitem{oop-book}
Arthur \textsc{Riel}
(1996).\\
\emph{Object-Oriented Design Heuristics}
(1ra ed.). Addison-Wesley.

\medskip
\bibitem{design-patterns-book}
\textsc{Gamma}~E., \textsc{Helm}~R., \textsc{Johnson}~R. y \textsc{Vlissides}~J.
(1994).\\
\emph{Design Patterns: Elements of Reusable Object-Oriented Software}
(1ra ed.). Addison-Wesley Professional.

\medskip
\bibitem{algo-book}
Steven \textsc{Skiena}
(2008).\\
\emph{The Algorithm Design Manual}
(2da ed.). Springer.

\medskip
\bibitem{unix-impatient}
Paul \textsc{Abrahams}
(1995).\\
\emph{UNIX for the Impatient}
(2da ed.). Addison-Wesley.

\medskip
\bibitem{pro-git}
Scott \textsc{Chacon}.\\
\emph{Pro Git}.\\
\url{https://git-scm.com/book}

\medskip
\bibitem{autotools}
John \textsc{Calcote}
(2010).\\
\emph{Autotools: A Practitioner's Guide to GNU Autoconf, Automake, and Libtool}
(1ra ed.). No Starch Press.

\medskip
\bibitem{code-complete}
Steve \textsc{McConnell}
(2004).\\
\emph{Code Complete: A practical handbook of software construction}
(2da ed.). Microsoft Press.

\medskip
\bibitem{gtk-doc}
\emph{GTK-Doc Manual}.\\
\url{https://developer.gnome.org/gtk-doc-manual/}

\medskip
\bibitem{gobject-introspection}
\emph{GObject Introspection}.\\
\url{https://wiki.gnome.org/Projects/GObjectIntrospection}

\medskip
\bibitem{gnome-programming-guidelines}
\emph{GNOME Programming Guidelines}.\\
\url{https://developer.gnome.org/programming-guidelines/stable/}

\end{thebibliography}

\end{document}